%------------------------------------------------------------------------------
%	ALIGNMENTS & MARGINS
%------------------------------------------------------------------------------

% margins
\usepackage[top=3.0cm, bottom=3.0cm,left=3.0cm,right=3.0cm]{geometry}

\usepackage{pdflscape,lipsum}
\usepackage{etoolbox}


\makeatletter
\def\ifGm@preamble#1{\@firstofone}
\appto\restoregeometry{%
  \pdfpagewidth=\paperwidth
  \pdfpageheight=\paperheight}
\apptocmd\newgeometry{%
  \pdfpagewidth=\paperwidth
  \pdfpageheight=\paperheight}{}{}
\apptocmd\loadgeometry{%
  \pdfpagewidth=\paperwidth
  \pdfpageheight=\paperheight}{}{}  
\makeatother

%------------------------------------------------------------------------------
%	HEADERS & FOOTERS
%------------------------------------------------------------------------------


\usepackage{fancyhdr}
\pagestyle{fancy}

% turn them into grey color
\usepackage{etoolbox}

\usepackage[x11names]{xcolor}

\definecolor{Grey}{RGB}{100, 100, 100}

\usepackage{blindtext}
\usepackage{microtype}
\usepackage[automark]{scrlayer-scrpage} 
\clearpairofpagestyles 

\setlength\parindent{0pt} % To turn off the automatic indentation

\ohead{\headmark} 
\rofoot*{% 
  \makebox[0pt][l]{%
     \hspace{\marginparsep}%  
     \raisebox{0pt}[\ht\strutbox][\dp\strutbox]{% 
      \rule[-\dp\strutbox]{1pt}{2\baselineskip}% 
     }% 
     \enskip 
     \pagemark 
  }% 
}
\lefoot*{% 
  \makebox[0pt][r]{% 
     \pagemark 
     \enskip
     \raisebox{0pt}[\ht\strutbox][\dp\strutbox]{% 
      \rule[-\dp\strutbox]{1pt}{2\baselineskip}% 
     }% 
     \hspace{\marginparsep}%
  }% 
}
\addtokomafont{pagehead}{\upshape}

% landscape
\usepackage{pdflscape}
\usepackage{capt-of}

%------------------------------------------------------------------------------
%	TITLES, SECTIONS & FONTS
%------------------------------------------------------------------------------

% change default font
\usepackage{cmbright}
\usepackage[OT1]{fontenc}

\usepackage[explicit]{titlesec}
\usepackage{textcase,relsize}

% \usepackage[T1]{fontenc}
\usepackage{microtype}
\usepackage{fourier}
\usepackage{cabin}
% \usepackage{titleps,kantlipsum}

\definecolor{ThesisGreen}{RGB}{128, 151, 139}


% no widow lines
\usepackage[all]{nowidow}

% control hyphenation
\pretolerance=1000
\tolerance=1000
\emergencystretch=0pt
\righthyphenmin=4
\lefthyphenmin=4

% lines spacing
\usepackage{setspace}
\setstretch{1.1}

% make HUGE font option
\usepackage{fix-cm}    

\makeatletter
\newcommand\HUGE{\@setfontsize\Huge{35}{45}}
\makeatother    

%------------------------------------------------------------------------------
%	MATH
%------------------------------------------------------------------------------

\usepackage{fancyhdr,graphicx,amsmath,amssymb,amsfonts,mathtools}
\usepackage{euscript} % Euler script
\usepackage{calrsfs} % Caligraphy in math

\DeclarePairedDelimiter{\nint}\lfloor\rceil % round to nearest integer symbol
\DeclareMathAlphabet{\pazocal}{OMS}{zplm}{m}{n} % Caligraphy in math
\setcounter{MaxMatrixCols}{20} % have more columns in matrices
\newcommand{\ccal}[1]{\pazocal{#1}}
\usepackage{ dsfont }

%------------------------------------------------------------------------------
%	ALGORITHM
%------------------------------------------------------------------------------

% \usepackage[table]{xcolor}% http://ctan.org/pkg/xcolor
%documentation: http://ctan.cs.uu.nl/macros/latex/contrib/xcolor/xcolor.pdf
\usepackage[
	linesnumbered,
	boxed,
	ruled,
	vlined
]{algorithm2e}
% algorithm comment style
\newcommand\mycommfont[1]{\scriptsize\ttfamily\textcolor{darkgray}{#1}}
\SetCommentSty{mycommfont}
% algorithm numbering style
\newcommand\mynumfont[1]{\footnotesize\ttfamily\textcolor{darkgray}{#1}}
\SetNlSty{mynumfont}{}{}
% set algorithm caption size
% \SetAlCapFnt{\footnotesize}
% set the algorithm font
% \SetAlFnt{\ttfamily}
%\removing the indent of algorithm
\setlength{\algomargin}{1.2em}
% create a while loop construct
\SetKwFor{While}{while}{do}{end}
% for setting the line space in algorithm
\usepackage{setspace}

\usepackage[hang,flushmargin]{footmisc}
\makeatletter
\newcommand{\algorithmfootnote}[2][\footnotesize]{%
  \let\old@algocf@finish\@algocf@finish% Store algorithm finish macro
  \def\@algocf@finish{\old@algocf@finish% Update finish macro to insert "footnote"
    \leavevmode\rlap{\begin{minipage}{\linewidth}
    #1#2
    \end{minipage}}%
  }%
}
\makeatother

%------------------------------------------------------------------------------
%	ITEMIZE & ENUMERATE OPTIONS
%------------------------------------------------------------------------------

\usepackage{enumitem}% http://ctan.org/pkg/enumitem
% setting the itemize line spacing for the whole document
\let\OLDitemize\itemize
% Custom enumerate (1)
\newlist{myEnum}{enumerate}{1}
\setlist[myEnum]{label=(\arabic*)}
\renewcommand\itemize{\OLDitemize[noitemsep,topsep=0pt]}
% setting the enumerate line spacing for the whole document
\let\OLDenumerate\enumerate
\renewcommand\enumerate{\OLDenumerate[noitemsep,topsep=0pt]}
% use roman numbers in text
\newcommand{\RNum}[1]{\uppercase\expandafter{\romannumeral #1\relax}}

%------------------------------------------------------------------------------
%	TABLES, FIGURES & CAPTIONS
%------------------------------------------------------------------------------

\usepackage{multirow,tabularx, booktabs} % advanced table package
\usepackage{makecell}
% \usepackage{tablefootnote} % add footnotes to tables
\usepackage{threeparttable} % add footnotes directly bellow table
\def\arraystretch{1.25} % 25 percent more vertical space for each row
\renewcommand{\tabularxcolumn}[1]{m{#1}} % align vertically to the middle

\usepackage{caption} % packages for subfigures/captions
\usepackage{subcaption}

\usepackage[table]{xcolor}
\usepackage{booktabs}

\usepackage[justification=centering, 
font=small, 
labelfont=bf, 
textfont=it]{caption} % package for adding and styling the captions


% placement of floats
\usepackage{float} % to affect the placement of elements
\usepackage{placeins} % to place barriers for float objects

\usepackage[demo]{graphicx}

% to create plots
\usepackage{tikz, pgfplots}
\usetikzlibrary{positioning}
\usetikzlibrary{positioning,fit,calc} % to draw outer nodes and fit the inner ones better

\definecolor{c1}{rgb}{1, 0.868852459, 0.504098361}
\definecolor{c2}{rgb}{0.874125874, 0.923076923, 1 }
\definecolor{c3}{rgb}{0.96, 0.975, 1}
\definecolor{c4}{rgb}{0.724867725, 0.920634921, 1}
\definecolor{c5}{rgb}{0.441176471, 0.788235294, 1}
\definecolor{c6}{rgb}{0.333333333, 0.609195402, 1}

% for cover background image
\usepackage{wallpaper}
\usepackage{eso-pic}
\newcommand\BackgroundPic{%
\put(0,0){%
\parbox[b][\paperheight]{\paperwidth}{%
\vfill
\centering
\includegraphics[width=\paperwidth,height=\paperheight]{Figures/Cover_back.jpg}%
\vfill
}}}

% macros for vertical line seperator
\usepackage[]{mdframed}
\newmdenv[
        topline=false,
        bottomline=false,
        rightline=false,
        linewidth=2pt,
        innerleftmargin=5pt,
        leftmargin=0pt,
        rightmargin=0pt,
        innerbottommargin=0pt
]{separator}

% macro for footnote without a marker
\newcommand\blfootnote[1]{%
  \begingroup
  \renewcommand\thefootnote{}\footnote{#1}%
  \addtocounter{footnote}{-1}%
  \endgroup
}

%------------------------------------------------------------------------------
%	BIBLIOGRAPHY & REFERENCES
%------------------------------------------------------------------------------

\usepackage{hyperref} % 
\hypersetup{
    colorlinks=True,
    urlcolor=cyan,
    linkcolor=,
    citecolor=gray,
    pdftitle={MSc Thesis - Christos Lathourakis},
    }
    
\usepackage[
    backend=biber,
    style=ieee
]{biblatex} % bibliography package and styling

% change font size of references
\renewcommand*{\bibfont}{\footnotesize}

%------------------------------------------------------------------------------
%	GLOSSARY - ACRONYMS
%------------------------------------------------------------------------------

\usepackage[acronym, style=super]{glossaries-extra}
\setabbreviationstyle[acronym]{long-short}
\makeglossaries

% Input the list of acronyms used in the report
\newacronym{DRL}{DRL}{Deep Reinforcement Learning}
\newacronym{DQN}{DQN}{Deep Q-Network}
\newacronym[longplural=Markov Decision Processes]{MDP}{MDP}{Markov Decision Process}
\newacronym{AI}{AI}{Artificial Intelligence}
\newacronym{BMU}{BMU}{Bayesian Model Updating}
\newacronym[longplural=Partially Observable Markov Decision Processes]{POMDP}{POMDP}{Partially Observable Markov Decision Process}
\newacronym[longplural=Semi Markov Decision Processes]{SMDP}{SMDP}{Semi Markov Decision Process}
\newacronym{DP}{DP}{Dynamic Programming}
\newacronym{RL}{RL}{Reinforcement Learning}
\newacronym{A2C}{A2C}{Advantage Actor Critic}
\newacronym{PPO}{PPO}{Proximal Policy Optimization}
\newacronym{TRPO}{TRPO}{Trust Region Policy Optimization}
\newacronym{SDOF}{SDOF}{Single Degree of Freedom}
\newacronym{HMC}{HMC}{Hamiltonian Monte Carlo}
\newacronym{NUTS}{NUTS}{No-U-Turn Sampler}
\newacronym{MCMC}{MCMC}{Markov Chain Monte Carlo}
\newacronym{OMA}{OMA}{Operational Modal Analysis}
\newacronym{DL}{DL}{Deep Learning}
\newacronym{DNN}{DNN}{Deep Neural Network}
\newacronym{VoI}{VoI}{Value of Information}
\newacronym{VoSHM}{VoSHM}{Value of Structural Health Monitoring}
\newacronym{SHM}{SHM}{Structural Health Monitoring}
\newacronym{DBN}{DBN}{Dynamic Bayesian Network}
\newacronym{SSI}{SSI}{Stochastic Subspace Identification}
\newacronym{FEA}{FEA}{Finite Element Analysis}
\newacronym{FEM}{FEM}{Finite Element Method}
\newacronym{FE}{FE}{Finite Element}
\newacronym{DDQN}{DDQN}{Double Deep Q-Network}
\newacronym{TD}{TD}{Temporal Difference}
\newacronym{TMCMC}{TMCMC}{Transitional Markov Chain Monte Carlo}
\newacronym{SMC}{SMC}{Sequential Monte Carlo}
\newacronym{DDMAC}{DDMAC}{Deep Decentralized Multi-agent Actor Critic}
\newacronym{DCMAC}{DCMAC}{Deep Centralized Multi-agent Actor Critic}
\newacronym{CNN}{CNN}{Convolutional Neural Network}
\newacronym{ANN}{ANN}{Artificial Neural Network}
\newacronym{RNN}{RNN}{Recurrent Neural Network}
\newacronym{DDPG}{DDPG}{Deep Deterministic Policy Gradient}
\newacronym{KL}{KL}{Kullback–Leibler}
\newacronym{HCRL}{HCRL}{Hierarchical Coordinated Reinforcement Learning}
\newacronym{CBM}{CBM}{Condition-Based Maintenance}
\newacronym{TBM}{TBM}{Time-Based Maintenance}
\newacronym{MC}{MC}{Monte Carlo}
\newacronym{CV}{CV}{Coefficient of Variation}
\newacronym{KDE}{KDE}{Kernel Density Estimation}
\newacronym{RC}{RC}{Reinforced Concrete}
\newacronym[longplural=Degrees of Freedom]{DOF}{DOF}{Degree of Freedom}
\newacronym{SLS}{SLS}{Serviceability Limit State}
\newacronym{FORM}{FORM}{First Order Reliability Method}
\newacronym{RVD}{RVD}{Random-Variable Degradation}
\newacronym{SPD}{SPD}{Stochastic-Process Degradation}
\newacronym{PDF}{PDF}{Probability Density Function}
\newacronym{CDF}{CDF}{Cumulative Distribution Function}
\newacronym{LSF}{LSF}{Limit State Function}
\newacronym{LSS}{LSS}{Limit State Surface}
\newacronym{ReLU}{ReLU}{Rectified Linear Unit}

%------------------------------------------------------------------------------
%	CUSTOM FUNCTIONS & MISCELLANEOUS SETTINGS
%------------------------------------------------------------------------------

\newcommand\red[1]{\textcolor{red}{#1}} % red text: texts that needs attention
\newcommand{\comment}[1]{} % multiline comment function
\newcommand{\prob}[1]{\text{P}(#1)}
\newcommand{\scnot}[1]{\mathrm{e}#1} % scientific notation
\newcommand{\myparagraph}[1]{\paragraph{#1}\mbox{}\\} % line break after paragraph

\newcommand\ddfrac[2]{\frac{\displaystyle #1}{\displaystyle #2}} % new fraction function to prevent the math symbols from getting crammed in a fraction


\usepackage{soul} % for strikethrough

\usepackage{cprotect}